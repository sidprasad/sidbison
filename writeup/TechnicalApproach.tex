\chapter{Implementing sidBison Commands}

\verb|sidBison| is designed to be a step-through debugger at the grammar level, and is built on top of \verb|iBison|'s interactive debugging mechanism. The mathematical underpinnings of the system involve mapping \verb|Bison| and \verb|iBison| constructs like 
\begin{enumerate}
\item Push-Down Automata
\item State Stacks
\item Token Stacks
\item Lookaheads
\end{enumerate}
\noindent
to Extended Backaus Naur Form (EBNF) grammar rules. As a result, each command takes the form of a mathematical function in terms of these variables. Thus, \verb|sidBison| commands can be thought of as a set $\{ f_i \}$ of functions.

\paragraph{step}

The \verb|step| command allows the user to step forward to the next action taken by the underlying parser debugger. It is the simplest command for the \verb|sidBison| system. The rule is of the form $f_1(ss, ts, ct) \rightarrow (ss', ts', tc')$, where:

\begin{enumerate}
\item $ss$ and $ss'$ are state stacks, where $ss'$ is a new state stack created by taking a step.
\item $ts$ and $ts'$ are token stacks, where $ts$ is a new token stack created by taking a step.
\item $ct$  and $ct'$ are current tokens, where $ct'$ is a new current token obtained either by a new lookahead or from $ss'$.
\end{enumerate}

\verb|step| provides the basis of mapping actions in terms of the underlying parser to those involving \verb|Bison| specifications. It maps a new \textit{lookahead} to encountering a new token, a shift to adding the token to the string, and a reduce to parsing a string. Several of the more complicated \verb|sidBison| commands leverage \verb|step|.

\paragraph{steprule}

The \verb|steprule| command takes the user to the next rule encountered by the parser. Thus, it presents a step-wise debugging abstraction in terms of the user-provided grammar specification instead of parser-generator options. 

The command is implemented by stepping through \verb|iBison| state changes till a reduce action is executed. As a result
it has the mathematical form $f_2: \verb|Parser Automaton| \rightarrow (ss`, ts`)$, where :

\begin{enumerate}
\item $ss'$ represents a new state stack, and by extension a new current state.
\item $ts'$ represents a new token stack.
\end{enumerate}

\paragraph{crule}

The \verb|crule| command returns the current rule being parsed by Bison generated parser. It takes the form of a mathematical function \\ $f_3: (\mbox{Parser Automaton},  $cs$  , $ss$) \rightarrow (\mbox{Backus Naur Form Rule})$, where $cs$ is the current state, and $ss$ is the state stack.\\

In general, a bottom-up parser cannot predict the non-terminal to which a partially known sequence of tokens will be reduced. Even judging if a reduction will ever take place is an undecidable problem. As a result, the command cannot be implemented using a single instance of a single-pass parser like \verb|Bison|. \verb|crule| therefore utilizes a time-travelling heuristic.

This approach is concurrent in nature. A secondary \verb|iBison| process is spawned and is placed in the same state as the primary one. This secondary debugger is then stepped forward to the first parser \textit{reduce} action where the state stack has either shrunk, or has the same size with a different top element. The reduced rule is precisely the production being parsed.

\paragraph{rulepos}

\verb|rulepos| enumerates current positions the parser may be in within grammar rules. It is implemented by examining the rules
and positions associated with the current rule in the underlying automaton. It has the mathematical form $f_4: cs \rightarrow [(\mbox{Backus Naur Form Rule},  p)]$, where $cs$ represents the current state and $p$ represents a position in the rule.

\paragraph{str}

The \verb|str| command returns the current position in the overall parsing process. It is implemented by displaying the contents of the \verb|iBison| token stack. It has mathematical form $f_5: cs \rightarrow str_{ss}$, where $cs$ represents the current state and $str_{ss}$ is a string representation of the state stack.

\paragraph{ctkn}

The \verb|ctkn| command displays the token with which the parser is currently dealing. It is implemented by presenting the newer of the look-ahead token and the top element of the token stack.


\paragraph{br}

The \verb|br| command allows the user to break when a particular token is encountered. This functionality is provided by stepping till the \verb|ctkn| equals the token provided. Thus it has mathematical form $f_6 : tkn \rightarrow (cs', ss')$, where:

\begin{enumerate}
\item $ss'$ represents a new state stack, and by extension a new current state.
\item $ts'$ represents a new token stack.
\item $tkn$ represents the input token.
\end{enumerate}


The \textbf{quit} and \textbf{test} commands are not involved in the actual debugging process, and so are not explained above.
